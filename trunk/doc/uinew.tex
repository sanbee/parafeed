\documentclass[11pt]{article}
\usepackage{html}

\setlength{\evensidemargin}{-0.7cm}
\setlength{\oddsidemargin}{-0.7cm}
\setlength{\textwidth}{17.8cm}
\setlength{\topmargin}{-1.cm}
\setlength{\textheight}{22cm}
\parskip 1ex    % White space between paragraphs amount
\renewcommand{\cite}      {\citep}
\newcommand{\sigmaD}      {{$\Sigma$-$D$}}
\newcommand{\sigmaDz}     {{$\Sigma$-$D$-$z$}}
\newcommand{\csq}         {{\it Central Square\ }}
\newcommand{\TTilde}      {{\char126}}
%\newcommand{\HTTP}        {{http://langur.ncra.tifr.res.in/$_\char126$sanjay/Offline}}
\newcommand{\OFFLINE}     {{http://langur.ncra.tifr.res.in/$_\char126$sanjay/Offline}}
\newcommand{\CATS}        {{http://www.ratan.sao.ru/\home cats}}
\newcommand{\CC}          {{cm$^{-3}$}}
\newcommand{\CMSIX}       {{cm$^{-6}$}}
\newcommand{\EMUNITS}     {{cm$^{-6}$~pc$^{-1}$}}
\newcommand{\arcmin}      {{^\prime}}
\newcommand{\arcsec}      {{^{\prime\prime}}}
\newcommand{\arcsecSq}    {{\mathrm{arcsec^2}}}
\newcommand{\arcminSq}    {{\mathrm{arcmin^2}}}
\newcommand{\degree}      {{^\circ}}
\newcommand{\RA}          {{\mathrm{RA}_{J2000}}}
\newcommand{\Dec}         {{\mathrm{Dec}_{J2000}}}
\newcommand{\home}        {{$_\char126$}}
%\newcommand{\home}       {{$\sim$}}
\newcommand{\dNe}         {{\Delta N_e}}
\newcommand{\la}          {{\langle}}
\newcommand{\ra}          {{\rangle}}
\newcommand{\xtract}      {{\tt xtract}}
\newcommand{\badbase}     {{\tt badbase}}
\newcommand{\rantsol}     {{\tt rantsol}}
\newcommand{\getamp}      {{\tt getamp}}
\newcommand{\getphs}      {{\tt getphs}}
\newcommand{\getres}      {{\tt getres}}
\newcommand{\closure}     {{\tt closure}}
\newcommand{\gltofit}     {{\tt gl2fit}}
\newcommand{\ltainfo}     {{\tt ltainfo}}
\newcommand{\ltacat}      {{\tt ltacat}}
\newcommand{\ltacleanup}  {{\tt ltacleanup}}
\newcommand{\calibbp}     {{\tt calibbp}}
\newcommand{\fixit}       {{\tt fixit}}
\newcommand{\offstop}     {{\tt offstop}}
\newcommand{\tmac}        {{\tt tmac}}
\newcommand{\lin}         {{\tt lin}}
\newcommand{\AIPS}        {{\tt AIPS}}
\newcommand{\record}      {{\tt record}}
\newcommand{\UNIX}        {{\tt UNIX}}
\newcommand{\CPP}         {{\tt C++}}
\newcommand{\C}           {{\tt C}}
\newcommand{\FORTRAN}     {{\tt FORTRAN}}
\newcommand{\LTAVIEW}     {{\tt LTAVIEW}}
\newcommand{\LTAFMT}      {{\tt LTAFMT}}
\newcommand{\LTARec}      {{\tt LTARec}}
\newcommand{\TABLE}       {{\tt table}}
\newcommand{\matmon}      {{\tt matmon}}
\newcommand{\oddix}       {{\tt oddix}}
\newcommand{\INDXR}       {{\tt INDXR}}
\newcommand{\SETJY}       {{\tt SETJY}}
\newcommand{\GETJY}       {{\tt GETJY}}
\newcommand{\UVPLT}       {{\tt UVPLT}}
\newcommand{\VPLOT}       {{\tt VPLOT}}
\newcommand{\EXTD}        {{\tt EXTDEST}}
\newcommand{\BPASS}       {{\tt BPASS}}
\newcommand{\FLGIT}       {{\tt FLGIT}}
\newcommand{\POSSM}       {{\tt POSSM}}
\newcommand{\SNPLT}       {{\tt SNPLT}}
\newcommand{\UVFLG}       {{\tt UVFLG}}
\newcommand{\TVFLG}       {{\tt TVFLG}}
\newcommand{\SPFLG}       {{\tt SPFLG}}
\newcommand{\CLCAL}       {{\tt CLCAL}}
\newcommand{\CALIB}       {{\tt CALIB}}
\newcommand{\SPLIT}       {{\tt SPLIT}}
\newcommand{\MSUN}        {{M_\odot}}
\newcommand{\FT}          {{\mathcal{FT}}}
\newcommand{\GOACS}       {GMRT on-line array control system}
\newcommand{\OHMASER}    {{OH (1720~MHz)}}
\newcommand{\XRAYCOMPOSITE}    {{X-ray-composite}}
\newcommand{\UCHII}       {{UC~H~II}}

\newcommand{\mla}         {\langle}
\newcommand{\mra}         {\rangle}
\newcommand{\Rp}          {{\it p}}
\newcommand{\Lp}          {{\it q}}
\newcommand{\RR}          {{pp}}
\newcommand{\LL}          {{qq}}
\newcommand{\EirNot}      {{E^\Rp_{i,\circ}}}
\newcommand{\EilNot}      {{E^\Lp_{i,\circ}}}
\newcommand{\EjrNot}      {{E^\Rp_{j,\circ}}}
\newcommand{\EjlNot}      {{E^\Lp_{j,\circ}}}
\newcommand{\CongjEjlNot} {{E^{\Lp\star}_{j,\circ}}}
\newcommand{\CongjEjrNot} {{E^{\Rp\star}_{j,\circ}}}
\newcommand{\Rgigj}       {{g^\Rp_i g_j^{\Rp^\star}}}
\newcommand{\Laiaj}       {{\alpha^\Lp_i \alpha_j^{\Lp^\star}}}
\newcommand{\REiEj}       {{E_i^\Rp E_j^{\Rp^\star}}}
\newcommand{\LEiEj}       {{E_i^\Lp E_j^{\Lp^\star}}}
\newcommand{\RoEiEj}      {{E_i^{\Rp^o} E_j^{{\Rp^o}^\star}}}
\newcommand{\LoEiEj}      {{E_i^{\Lp^o} E_j^{{\Lp^o}^\star}}}
\newcommand{\sumij}       {\sum\limits_{{i,j} \atop {i \ne j}}}
\newcommand{\sumj}        {\sum\limits_{j \atop {j \ne i}}}
\newcommand{\Xij}         {X_{ij}}
\newcommand{\wij}         {w_{ij}}
\newcommand{\wijRR}       {w_{ij}^{\Rp\Rp}}
\newcommand{\wijLL}       {w_{ij}^{\Lp\Lp}}
\newcommand{\XijRR}       {X_{ij}^{\RR}}
\newcommand{\giR}         {{g_i^\Rp}}
\newcommand{\giL}         {{g_i^\Lp}}
\newcommand{\aiL}         {{\alpha_i^\Lp}}
\newcommand{\aiR}         {{\alpha_i^\Rp}}
\newcommand{\gjR}         {{g_j^\Rp}}
\newcommand{\gjL}         {{g_j^\Lp}}
\newcommand{\ajL}         {{\alpha_j^\Lp}}
\newcommand{\ajR}         {{\alpha_j^\Rp}}
\newcommand{\gv}          {{\bf \vec g}}
\newcommand{\av}          {{\bf \vec \alpha}}
\newcommand{\SelfCal}     {{self-calibration}}
\newcommand{\antsol}      {{\tt antsol}}
\newcommand{\coPolar}     {{\it co-polar}}
\newcommand{\crossPolar}  {{\it cross-polar}}
\newcommand{\LeakyAntsol} {{\tt leaky~\antsol}}
\newcommand{\rhoRRo}      {{\rho_{ij,\circ}^{\Rp\Rp}}}
\newcommand{\rhoRR}       {{\rho_{ij}^{\Rp\Rp}}}
\newcommand{\rhoLLo}      {{\rho_{ij,\circ}^{\Lp\Lp}}}
\newcommand{\rhoLL}       {{\rho_{ij}^{\Lp\Lp}}}

\newcommand{\HTTP}        {{http://www.aoc.nrao.edu/$_\char126$sbhatnag/GMRT\_Offline/offline.html}}
\newcommand{\x}{{\bf{x}}}
\newcommand{\xp}{{{\bf x}_p}}
\newcommand{\xpi}{{{\bf x}_p^i}}
\newcommand{\bu}{{\bf{u}}}
\newcommand{\arcsper}{\ifmmode \rlap.{'' }\else $\rlap{.}'' $\fi}
\newcommand{\arcs}{\ifmmode {'' }\else $'' $\fi}
\newcommand{\arcm}{\ifmmode {' }\else $' $\fi}

\begin{document}
\rm

\title{\bf The User Interface}
\author{S.Bhatnagar}
\date{Jan. 23, 1997}
\maketitle
\begin{center}
\htmladdnormallinkfoot{[Postscript version]}{http://www.aoc.nrao.edu/~sbhatnag/GMRT\_Offline/ui.ps}
\end{center}


\normalsize
\rm

\section{Introduction}

This document describes the user interface of the offline data
analysis package.  The motivation behind developing this user
interface system is to have a simple, uniform interface for all the
programs that we will write.

All applications using this system, when started, will start an
interactive shell which will present the list of keywords to which the
application is sensitive.  The user can then set/reset the values of
these keywords before starting the application.  The initial values of
any of these keywords can also be supplied as command-line arguments
in the format {\tt <KeyWord>=[<Val0>[,<Val1>[...]]]}, or loaded from a
file by a single command.

Numeric values of any of the keywords can be any algebraic expression.



\section{The User Interface Commands}

Upon startup, applications will present a list of parameters to which it
is sensitive.  This will be in the form of various keywords.  The
values of all these keywords will be empty, unless the keywords was
set on the command-line in the format
{\tt <KeyWord>=[<Val0>[,<Val1>[...]]]}.  

Following are the commands available from this interactive shell:

\begin{verbatim}

 cd       : Change working directory
 inp      : To see the various keywords and their values
 go       : To run the application
 gob      : To run the application in background
 help     : This help
 ?        : Information on the type of the keywords
 edit     : Use an editor to (un)set values
 save <F> : Save the values in the file <F>
 load <F> : Load the values from the file <F>
 explain  : Help on the meaning of the various keywords
 quit     : Quit the application

\end{verbatim}

These commands form the standard set of shell commands that all
applications will normally understand.  There may be a few extra
commands which may be application dependant and will change from
application to application.  However, the ``help'' command is always
expected to show all the commands that the application can understand.

The keywords can be set/reset from this shell by commands of the type
{\tt <KeyWord>=<Val>}.  For recognizing the names of the keywords in
such commands, the shell uses the minimum match algorithm - minimum
\label{UI_CMDS}

The user interface of all application programs by default presents a
list of keywords to which the application is sensitive.  This is
presented as a list of variables of an embedded interactive shell.  The
user can set, reset, load, save, etc. the values of these keywords
using the basic shell commands in an interactive shell.  In general,
keywords can be assigned a list of comma-separated values.  The number
of values that a keyword expects can be found via the shell-command
{\tt '?'}.  The value of a keyword can be set by a command of type
{\tt <KeyWord>=[<Val0>[,<Val1>[...]]]}  and can be reset by omitting
the values in such commands (command {\tt <KeyWord>=}).  The
shell-command {\tt help} provides the basic help about the shell
itself and prints following on the screen:

\begin{verbatim}

   Commands in the interactive mode:

   Use <Key>=<Val1,Val2,..> to set value(s) for a keyword
   Use <Key>=<RETURN> to unset value(s) for a keyword

    inp         : To see the various keywords and their values
    go          : To run the application
    gob         : To run the application in background
    cd          : Change working directory
    help        : This help
    ?           : Information on the type of the keyword
    explain     : Detailed help, optionally of keywords/task [[Key][:Task]]
    save        : Save the values, optionally in a file
    load        : Load the values, optionally from a file
    edit        : Use an editor to (un)set the values
    quit        : Quit the application

   Any other input will be passed to the system shell

\end{verbatim}

These commands form the standard set of shell-commands available from
all applications.  However, the user interface libraries provide
mechanisms for application programmers to add application specific
shell-commands as well as alter the behavior of the basic
shell-commands.  Hence, there may be a few extra commands which
may be application dependent and will change from application to
application.  However, the {\tt help} command will automatically
always show all the commands available for a given application.

Following is a short description of the shell commands:

\begin{itemize}

\item Command {\tt inp} displays the current setting of the keywords.
  
\item Command {\tt go} signifies the end of the interactive session.
  After this commands is issued, the values of the keywords set by the
  users are used by the application internally and the execution of
  the main program proceeds.
  
\item Command {\tt gob} is similar to {\tt go} except that it will run
  the application in the background and print the process ID (PID)
  number of the background process on the screen (also see section
  \ref{CUSTOMIZATION}).
  
\item Command {\tt cd} changes the current working directory within
  the interactive shell.
  
\item Command {\tt '?'} gives some basic information
  about the type and the number of values the keywords expects.
  
\item Command {\tt explain} gives as detailed a help
  about the application and the keywords as the author of the
  application has cared to write in the help file.  These help files
  are located in the directory specified by the environment variable
  {\tt GDOC} (see section \ref{CUSTOMIZATION}).
  
\item Command {\tt save} saves the current setting of the keywords in
  a file.  By default, these values are saved in a file named {\tt
    ./<Application Name>.def}.  If a file name is supplied as an
  argument of this command, the values will be saved in the file.
  this command.
  
\item Command {\tt load} is the conjugate of {\tt save} - by default
  it loads the settings for the keywords from the file {\tt
    ./<Application Name>.def}.  If this file is already present when
  the application is started, it is loaded automatically.
  Alternatively, it can load settings from a file provided as an
  argument to the {\tt load} command.
  
\item Command {\tt edit} allows user to edit the keyword values in an
  editor of choice specified by the environment variable {\tt EDITOR}
  (see section \ref{CUSTOMIZATION}).
  
\item Command {\tt quit} quits the interactive session without
  executing the application.

\end{itemize}

All inputs to the interactive shell, which are not any of the above
mentioned commands or any of the application specific commands, are
passed to the underlying Operating System (OS) user shell.  Hence,
most of the native OS commands shall still be available from this
shell.  (Users however must be aware that certain OS shell commands
like ``{\tt setenv}'' (for {\tt csh} users) or ``{\tt export}'' (for
{\tt bash/sh} users) will seem to work, but will not have the desired
effect).

As mentioned earlier, the values of the keywords can be a list of
comma-separated values.  The expected number of values will be
reported by the shell-command {\tt '?'} along with the type of the
values, enclosed in '{\tt [ }' and '{\tt ] }' pairs.  A blank set of
square brackets (``{\tt []}'') will be printed for keywords which can
accept any number of values.  Some keywords may accept values of mixed
types (e.g., string, floats, integers).  For such keywords, the
reported type will be ``{\tt UNKNOWN}''.

Comma is treated as a separator in a list of values for a keyword.  To
suppress its interpretation as a separator, it can to be ``escaped''
using the backslash ('$\backslash$') before the comma.  For example,
if the value of a keyword {\tt key} has to be set to a string {\tt
``Funny, value with a comma''}, it can be done using

\begin{verbatim}
                key=Funny\, value with a comma
\end{verbatim}

The characters '$\backslash$', '{\tt [}', '{\tt ]}', '{\tt =}', which
are part of the shell-command syntax, must also be escaped similarly
to be used as part of the values of the keywords.

Numeric values of keywords can be arbitrary mathematical expressions.
Following functions and constants can be used in these expression:

\begin{itemize}
\item Functions

sin,cos,tan,asin,acos,atan,atan2,sinh,cosh,tanh,\\
exp,ln,log,flog10,sqrt,fabs,floor,ceil,rint


\item Constants
\begin{enumerate}

\item {\tt PI:} value of $\pi$
\item {\tt C:} speed of light in meters per second.
\item {\tt R2D,D2R:} multiplicative constants for conversion from
Radians to Deg. and vice versa
\item {\tt H2R,R2H:} multiplicative constants for conversion from
Hours to Radians and vice versa
\item {\tt SOL2SID,SID2SOL:} multiplicative constants for conversion
from Solar to Sidreal time and vice versa
\end{enumerate}
\end{itemize}

Numbers in the expression can be in any of the following
representations:

\begin{itemize}
\item Integer format, real (float) format

\item 1.0E-1 or 1.0e-1 (=0.1)
        
\item 1.0D-1 or 1.0d-1 (=0.1)
        
\item 1h10m0.1s: the time format - converts the number to seconds
before using it.
\item $1d1\arcmin1\arcsec$: The angular format - converts to arc
seconds before using it.

($1d1\arcmin = 3660.0\arcsec$ and $-1d1\arcmin = -3660.0\arcsec$)
\end{itemize}


\subsection{De-referencing mechanism}
\label{DOLLOR_MECH} 

The user interface system allows the application programmers to define
shell symbols which are treated as constants (i.e. their values cannot
be modified by the user).  These symbols are referred to as the {\tt
  const symbols}.  Some applications may load a number of frequently
used values of various keywords as {\tt const symbols} (e.g.,
the list of symbols reported by the {\tt showfmt} command of the
\htmladdnormallinkfoot{{\tt xtract}}{http://www.aoc.nrao.edu/~sbhatnag/GMRT\_Offline/offline.html} program).  However,
by default, none of the {\tt const symbols} are presented to the user
and its the responsibility of application program to add extra
shell-commands for viewing the list of {\tt const symbols}.

Values can be transfered from one keyword or a {\tt const symbol} to
another keyword.  E.g. to transfer the $i^{th}$ value from keyword
{\tt Key1} to a keyword {\tt Key2}, one could use {\tt Key2 =
\$Key1[i]}.  {\tt Key2=\$Key1} will transfer all values of {\tt Key1}
to {\tt Key2}.


\subsection{File name convention}

Most off-line applications can perform I/O using UNIX pipes.  If the
input file name begins with '{\tt <}', the rest of the file name is
treated as a command, the output of which becomes the input of the
application. For example, to supply the output of a program \tmac\ as
the input of the program \xtract, the {\tt ``{\tt in}''} keyword of
\xtract\ must be set to `{\tt <tmac}''.  After the shell-command {\tt
go} is issued for \xtract, the interface of \tmac\ will be started and
the keywords of \tmac\ can then be set in the normal fashion.  When
the shell-command {\tt go} is issued again, \tmac\ will begin
execution, and the output of \tmac\ becomes the input of \xtract.
Till this time, execution of \xtract\ would remain blocked, waiting
for input from \tmac.

Similarly, if the output file name begins with the '{\tt |}'
character, the rest of the file name is treated as a command, the
input of which becomes the output of the off-line program.

By convention, a blank output file name implies that the output would
go to the standard output stream (usually your screen) and a blank
input file name implies that the input would be read from the standard
input stream (usually your keyboard).

\section{Customization}
\label{CUSTOMIZATION}

The user interface can be customized using the following environment
variables. 

\begin{itemize}

\item        {\tt GDOC}
  
  {\tt GDOC} must point to the standard directory where documents
  explaining the various off-line data analysis programs are kept.
  The {\tt explain} command will first look for the explanation file
  in the local directory and then in the directory specified by this
  variable.  The name of the explanation file is constructed by
  appending the suffix {\tt ``.doc''} to the application name.

\item        {\tt GERR, GOUT}
  
  {\tt GERR} and {\tt GOUT} variables are used when the application is
  run using the {\tt gob} command (see section \ref{UI_CMDS}).  The
  standard output of the application will be redirected to the file
  specified by {\tt GOUT} while the standard error stream will be
  redirected to the file specified by {\tt GERR}.  By default, these
  variables are set to {\tt /dev/null}.

\item   {\tt GCONF}
  
  Some applications may load frequently used setting for some
  keywords.  These values are loaded as constants of the interactive
  shell and can be stored in a configuration file.  Their values can
  be transfered to the application keywords by referring to their
  value by '\$' mechanism (see section \ref{DOLLOR_MECH}).  The path
  of the directory containing these configuration files is specified
  by the environment variable {\tt GCONF}.  The configuration filename
  is constructed by appending the suffix ``{\tt .config}'' to the
  application name.  If {\tt GCONF} is not defined, the application
  looks in the directory specified by {\tt GDOC}.  If this variable is
  also not defined, or the configuration file is not found, the
  application will look for the configuration file in the current
  directory.

\item   {\tt EDITOR}
  
  {\tt EDITOR} environment variable is used to specify the name of the
  text editor to be used in the {\tt edit} command (see section
  \ref{UI_CMDS}).  The default editor is {\tt emacs}.

\item   {\tt GDEFAULTS}
  
  The default values of keywords can be saved in ``defaults file''.
  {\tt GDEFAULTS} variable specifies the directory where these files
  are stored.  Such a file can be automatically loaded by the
  applications upon startup.  The defaults file name is constructed by
  appending the suffix ``{\tt.def}'' to the application name.
  
  By default, the application will look for the defaults file in the
  current directory.  If a keyword appears in the ``{\tt.def}'' as
  well as in the ``{\tt .config}'' file, the keyword will be treated
  as a shell constant.  This can be used to effectively produce
  specialized versions of an application program by writing an
  appropriated ``{\tt .config}'', where keywords can have fixed
  values, not alterable by the user (e.g., a version of {\tt
    xtract} which will read input from the shared memory of the GMRT
  data acquisition system).

\end{itemize}

The following two variables are effective only for versions of the
user interface libraries which use the GNU Readline and the History
libraries.

\begin{itemize}
\item   {\tt GHIST}
  
  {\tt GHIST} specifies the file in which the history of the commands
  issued in the interactive session is saved.  This file will be
  common to all applications.  The default history file is {\tt
    \$HOME/.g\_hist}.
  
  This is also the file from which all applications will load the
  command history.

\item   {\tt MAXGHIST}
  
  {\tt MAXGHIST} should be set to the maximum number of command
  history entries which the user wishes to save.  By default this is
  set to 100.
\end{itemize}


\subsection{The {\tt help} keyword}

All applications are sensitive to the keyword {\tt help} (not to be
confused with the shell command {\tt help} described in section
\ref{UI_CMDS}).  It is never displayed in the list of keywords.  To
use it, it must be specified as a command-line argument.  Following is
the list of values that can be set for this keyword on the
command-line and their effect:

\begin{itemize}

\item {\tt help=noprompt}
  
  The application runs in the non-interactive mode.  This is useful
  when the application is run from a shell script.
  
  When run in this mode, value of all the keywords which needs to be
  set must be supplied on the command-line (the order of the
  command-line options is not important).  The keywords must be fully
  spelled in the command-line options (i.e., no minimum-match will be
  applicable).

\item {\tt help=explain}
  
  This executes the {\tt explain} command of the interactive shell
  (see section \ref{UI_CMDS}) without starting the interactive shell.
  
  If a keyword is supplied within brackets ({\tt '('} and {\tt ')'})
  immediately after the {\tt explain} string (e.g. {\tt
    help=explain(out)}, help will be provided for the specified
  keyword alone.  If an application name is also included within the
  brackets, separated from the keyword by a colon ('{\tt :}') (e.g.
  {\tt help=explain(out:xtract)}, help for the named keyword of the
  named application will be provided.  If the keyword is skipped (but
  not the colon), entire help of the named application will be
  provided.

\item {\tt help=doc}
  
  This results into an empty documentation file written on the
  standard output, in the required format, with a list of keywords to
  which the application is sensitive.  This is for use by authors of
  the applications and to encourage them to not only write the
  documentation, but also in a uniform format.

\item {\tt help=dbg}
  
  Application programmers can define hidden keywords which are
  normally not displayed for the user, but will be used by the
  application internally.  These keywords can be accessed as normal
  keywords by setting {\tt help} keyword to the value ``{\tt dbg}''.

\end{itemize}

\end{document}
